% -------------------------------------------------------------
% -------------------------------------------------------------
\section{Introduction}
% -------------------------------------------------------------
\subsection{Work objectives}
Why is ITER so large? How have been determined its main dimensions? Are these dimensions sufficient to reach its scientific goals,  i.e. produce 500 MW of fusion power for pulses of 400s with a power amplification ratio of 10\footnote{\url{https://www.iter.org/sci/Goals}}? And what should be the size of tokamak fusion reactor: smaller or larger than ITER? How much additional power is required to reach positive power amplification? How does it affect the magnets and the plasma facing components? 

Answering these questions is the main motivation for this group work. During the few days dedicated to this work, you will have to answers some of these questions by working in small groups. Interactions between groups is highly encouraged: no one can be expert of all the various physical and technological fields required to build a fusion machine, so group work is essential. When necessary, results will be shared between groups.

In the first part of the work, you will derive from a 0-dimensional approach the necessary equations to deduce a couple $(R_0, B_0)$ for a given reactor project of prescribed $(P_{DT}, Q)$. By proceeding this way, some other characteristics of the plasma (namely the shape of the plasma, including elongation and aspect ratio, the average ion mass $\hat M$, the edge safety factor $q_a$ and the density $n_N$ normalised to the Greenwald density $n_G$) will have to be prescribed by other considerations. This method will be applied to deduce the ITER parameters $(R_0, B_0)$. 

Each group results will be presented and shared with each others during the Part 1 evaluation. Once done, you will have to fix the main dimensions of the tokamak to be build. In the Part 2, groups will then focus on particular aspects of the machine: magnets, MHD stability, plasma facing components, safety and remote handling, external heating systems, etc. In case of problems, questions or design change requests, groups will be able to exchange between each other, or even to organise a general meeting in which topics affecting all groups can be discussed.

% -------------------------------------------------------------
\subsection{Schedule (2019)}
\begin{itemize}
\item Thursday and Friday 7 and 8th of March: Part 1
\item Monday 11th of March: groups restitution
\item From Tuesday 12th to Thursday 14th: Part 2
\item Friday 15th of March: Final groups restitution
\end{itemize}

% -------------------------------------------------------------
% -------------------------------------------------------------
\section{Derivation of the Governing Relations}
The objective of this work is to deduce the dimensioning parameters of a tokamak machine, namely the large radius $R_0$ and the magnitude of the (toroidal) magnetic field $B_0$ (on-axis, i.e. at normalized radius $\rho=0$) for a given couple of fusion power $P_{fus}$ and amplification factor $Q$. Since we are seeking two parameters, we will try to develop a 0D-approach sufficient to get two independent equations starting from the definition of $P_{fus}$ and $Q$. 

% --------------------------------------------
\subsection{Preamble on Units}
The International System of Units (SI) should be used. Quantities which are \emph{not} expressed in SI units should be emphasised with a specific symbol, for example with hats ``$\hat{...}$''. When possible, try to combine all numerical values (coming from physical constants) into a single constant $C$.

% --------------------------------------------
\subsection{Fusion Power $P_{fus}$}
\subsubsection{Fusion reactivity}
To get nuclear fusion, nuclei have to come close enough to each other. Nuclear forces can overcome their mutual electrostatic repulsion provided their distance becomes sufficiently close, in the order of 1 Fermi ($10^{-15}$m). %This would require temperatures of the order of 720 keV for head-on collisions of thermal particles to lead to fusion reactions in a classical way. Actually, quantum physics has to be taken into account in the process. 

However, both in tokamaks and in stars interiors, fusion reactions take place predominantly due to the tunnel effect. Crossing this barrier can be quantified in a probabilistic manner with the reaction rate $R$ $[\mathrm{reaction/m^3 s}]$, defined as the probability of reaction per unit time and volume. 

The reaction rate between mono-energetic ions of density $n_1$ $\mathrm{[m^{-3}]}$ striking target ions of density $n_2$ $\mathrm{[m^{-3}]}$ is proportional to the effective cross-section area $\sigma$ $\mathrm{[m^2]}$ and to the velocity difference $v_{12}$ between the two species:

\begin{equation*}
r_{12} = n_1 n_2 \; \sigma v_{12}
\end{equation*}

The quantity $\sigma v_{12}$, which depends on the kinetic energy of the colliding particles, is called the reactivity ($\mathrm{[m^3/s]}$). Note that the reaction rate $r_{12}$ is proportional to the square of the density of the mixture. 

In fusion plasmas, ions are not mono-energetic. They are assumed to have Maxwellian velocity distributions. The average reactivity $\langle \sigma v \rangle_{12}$ derives from the following expression:

\begin{equation*}
\left < \sigma v \right >_{12} 
= \int_{-\infty}^{+\infty} \int_{-\infty}^{+\infty} 
\sigma(v_{12}) v_{12}\;  f_1(v_1) f_2(v_2) \; dv_1dv_2
\end{equation*}

Finally, the average reaction rate $R_{12}$ reads:

\begin{equation*}
R_{12} = n_1 n_2 \; \left < \sigma v \right >_{12}
\end{equation*}

It governs the time evolution of both densities: $dn_1/dt = dn_2/dt = -R_{12}$.


The reactivity depends of the temperature and its evolution is known from the measurements of the cross-sections. The temperature dependence of the reactivity $\langle \sigma v \rangle_{12}$ is plotted on figure \ref{fig:reactivity} for several fusion reactions (interpolated data from \cite{Huba2013}). 

\begin{figure} 
    \begin{center}
        \includegraphics[width=0.75\textwidth]{figures/reactivity_DT.png}
        \caption{Fusion reactivity }
        \label{fig:reactivity}
    \end{center}
\end{figure}

Of course, there is interest in operating a fusion reactor at the lowest temperatures. Looking to the figure \ref{fig:reactivity}, the mixture D-T seems the "easiest" way to initiate fusion reactions, since the D-T reaction  has the maximum reactivity at lowest temperature of all fusion reactions. The D-T reactivity reaches its maximum for a temperature of 64 keV, corresponding to a temperature of $742\,10^6$ K. The D-T reaction is \cite{FusionCEA1987}: 

\begin{equation*}
    \mathrm{D + T} \longrightarrow \mathrm{{}^4 He~(3.56~MeV) + n~(14.03~MeV)}
\end{equation*}

The D-T reaction leads to a total released energy of $E_{DT}$ = 17.59 \si{MeV} = $2.82\times 10^{-12} \si{J}$ per fusion reaction\footnote{This value can be compared to the 200 MeV released by $^{235}$U fission. Yet, the energy release \emph{per nucleon} ($i.e.$ per kilogram) is approximately 4 times larger for fusion than for fission reactions.}. Notice that the ratio of the total energy to that carried by the alpha particles $\lambda \doteq 17.59/3.56 \approx 4.94$ is not exactly equal to 5 as one would expect on the basis of momentum conservation. Actually, this ratio \emph{is obviously} consistent with momentum conservation, as it should, provided one takes into account relativistic effects. This point is detailed in section \ref{appendix:fusion_power}.

The fusion power per unit volume $p_{DT}$ produced by the fusion of the nuclei of deuterium and tritium reads: 
\begin{equation*}
  p_{DT} = n_D n_T \left< \sigma v \right>_{DT} E_{DT}
\end{equation*}
with $n_D$ and $n_T$ the deuterium and tritium density and $\left< \sigma v \right>_{DT}$ the D-T reactivity. Assuming equal deuterium and tritium densities:
\begin{equation*}
  n_D = n_T = \frac{n}{2}
\end{equation*}
with $n$ the electron density, then the thermonuclear power density is:
\begin{equation*}
  p_{DT} = \frac{1}{4} n^2 \left< \sigma v \right>_{DT} E_{DT}
\end{equation*}

Assuming a constant reactivity in the plasma ("flat profile hypothesis") and using the tore volume $V$, the fusion power is: 
\begin{equation}
  P_{fus} = \frac{V}{4}
    n^2 \left< \sigma v \right>_{DT} E_{DT}
\end{equation}

% --------------------------------------------------
\subsubsection{Tore Volume}

We recall that the volume of a tore can be approximated\footnote{The expression is exact for $\kappa=1$.} by 

\begin{equation}
    V_t = 2\pi^2 \kappa R a^2
\end{equation}

\noindent
with $a$ the minor radius and $R$ the major radius of the tore. Here, $\kappa=b/a$ stands for the elongation of the plasma, $b$ being the largest radius of the ellipsoid. $\kappa=1$ for a circular cross-section, and is larger than unity for an elongated plasma.

Introducing the inverse aspect ratio $\varepsilon$ defined as $\varepsilon  \doteq a /R$, the volume then reads:

\begin{equation}
    \boxed{
    	V_t = C_V \varepsilon^2 R^3
    }
    \label{eq:tore_volume}
\end{equation}
\noindent
with $C_V = 2\pi^2$ and $\kappa=1$\footnote{Later eventually, $\kappa$ could be varied through the constant $C_V$ }. 

% --------------------------------------------------
\subsubsection{Relationship 1}
Using the tore volume eq.\ref{eq:tore_volume}, one readily obtains the fusion power:

\begin{equation}
\boxed{
	P_{fus} = \frac{C_V}{4}
  				\varepsilon^2 R^3
				n^2 \left< \sigma v \right>_{DT} E_{DT}
	}
	\label{eq:FusionPower1}
\end{equation}

This is the first equation relating $P_{fus}$ to the geometrical parameter $R$. However, it is not usable yet, since from figure \ref{fig:reactivity}, the reactivity $\left< \sigma v \right>$ clearly depends of the temperature. It will be thus necessary for the following to define the temperature at which the machine should operate. This will be part of the physical constraints discussion in the next section. 

% --------------------------------------------------
% --------------------------------------------------
\subsection{Amplification factor $Q$}
\subsubsection{Power Balance}
We remind that at equilibrium, sources terms equal sink terms: heating power $P_H$ equal losses $P_{loss}$
\begin{equation}
	P_H = P_{loss}
	\label{eq:power_balance_general}
\end{equation}
where the losses term will be discussed later.


To the contrary with neutrons which leave the plasma, charged $\alpha$ nuclei are confined by the magnetic field and should ideally transfer their energy to the main ions before being extracted\footnote{There are basically 2 ways for this energy transfer. Since the collision frequency scales like the velocity difference between the colliding species to the power $-3$ ($\nu_{coll,ss'}\sim n_{s'}/\Delta v_{ss'}^3$), alpha particles transfer their energy dominantly to the electrons, which are much faster due to their low inertia. Then two routes are possible for the energy transfer from the electrons to the ions. Either via collisions, or via turbulence. In the latter case, the mediator are the electrostatic plasma waves. The relative weight of those two channels is still a matter of research.}. The total -- or \emph{net} -- heating power $P_H$ is the sum of the auxiliary plasma heating (including Ohmic heating) $P_{ext}$ and of alpha heating $P_\alpha$:

\begin{equation}
P_{H} \doteq P_\alpha + P_{ext}
\label{eq:definition_net_power}
\end{equation}

\noindent
Hence, 
\begin{equation}
	P_\alpha + P_{ext} = P_{loss}
	\label{eq:power_balance_simplified}
\end{equation}

\subsubsection{Auxiliary Heating Power: Definition of $Q$}
The plasma amplification factor is defined as the ratio of the fusion power $P_{fus}$ to the external heating power supplied $P_{ext}$\cite[p.12]{Wesson2004}:

\begin{equation}
	Q \doteq \frac{P_{fus}}{P_{ext}}
\label{eq:definition_Q}
\end{equation}

Importantly, notice that $Q$ \emph{does not} encompass -- by far -- the entire question of the energetic efficiency of a fusion reactor. Indeed, in particular, it does neither account for the energy used for cryogenic purposes (as required by the use of superconductors) nor for the conversion factor of thermal to electric energy\footnote{This point is often misleading for the public and it is important to be factually correct. Misleading statements concerning fusion and ITER power and energy since decades have been highlighted in \url{http://news.newenergytimes.net/2017/10/06/the-iter-power-amplification-myth/}. This led directly ITER to update its public information web pages on the Q-factor (cf. \url{https://www.iter.org/newsline/-/2845}).  Cf. also the follow-up \url{http://news.newenergytimes.net/2017/12/11/evidence-of-the-iter-power-deception/}}.




% --------------------------------------------------
\subsubsection{Thermal Losses: Energy Confinement time $\tau_E$}

It the following, radiations losses are neglected (at first). 

The energy confinement time $\tau_E$ is usually defined as the characteristic time at which this energy is lost from the plasma due to thermal transport\footnote{This definition excludes radiative losses. It is consistent with the one used for the ITER scaling laws, discussed later.}, either by collisional conduction or by turbulent thermal convection. Defining the energy confinement time as the rate of energy loss ($P_{loss}$)\cite[p.9]{Wesson2004}:

\begin{equation}
	P_{loss} \doteq \frac{ W }{ \tau_E } 
	\label{eq:definition_confinement_time}
\end{equation}

Experimental results have shown that the energy time in ELMy H-mode tokamak plasmas (referred to as the ITER IPB98(y,2) scaling law \cite[eq.(20)]{ITERphysics_chap2}) is well represented -- the root means square error is about 15.6\% -- by the following scaling law:

\begin{equation}
	\tau_E = C_{SL} \hat M^{0.19} \kappa^{0.78} \varepsilon^{0.58} 
	\hat n^{0.41} \hat I_p^{0.93} R^{1.97} B^{0.15}  \hat P_{net}^{-0.69}
	\label{eq:scaling_law_IPB98(y,2)}
\end{equation}

\noindent
with $C_{SL} = 0.0562$.


Here, $\hat I_p$ and $B$ are the plasma current and the toroidal magnetic field at the magnetic axis, respectively, $\hat n$ the line-averaged density, and $\hat M$ is the average ion mass (in Atomic Mass Unit)\footnote{Cf. also \url{http://fusionwiki.ciemat.es/wiki/Scaling_law}.}. 


It is important to realize that the exponents of the power law are actually bound by critical relations, which derive from the invariance properties of the underlying governing equations, namely Vlasov and Maxwell's equations. These relations are sometimes called the ``Kadomtsev constraints'', acknowledging his decisive contribution \cite{Kadomtsev1975} to import in the field of fusion science the pioneering works of Buckingham on scale invariance \cite{Buckingham1914}\footnote{On the question of dimensional analysis applied to physics in general, see also reference \cite{Misic2010}, and references \cite{Connor1977, Luce2008} for the specific case of fusion plasmas.}. For instance, as recalled in \cite{ITERphysics_chap2} (p.2204), a standard constraint imposes the relation 
$4\alpha_R - 8\alpha_n - \alpha_I - 3\alpha_P - 5\alpha_B = 5$, with $\alpha_X$ the exponent of variable $X$ in the scaling law. The rationale is briefly recalled in Appendix~\ref{appendix:scaling_law_dimensionless}.



% --------------------------------------------------
\subsubsection{Plasma Energy $W$}
The total internal energy of the plasma reads as follows:
\begin{equation*}
W  = \int \frac{3}{2} k_B \left( n_e T_e + n_i T_i \right ) dV 
\approx \int 3 n k_BT dV
\end{equation*}

\noindent
where the integral is performed over the plasma volume. Here, equal ion and electron temperatures have been assumed. Assuming flat density and temperature profiles and using the expression of the volume of a torus eq.\ref{eq:tore_volume}, the total plasma internal energy then reads (in [MW.s]):

\begin{equation}
\boxed{
	\hat W = C_{loss} C_V \varepsilon^2  \hat n \hat T R^3
	\label{eq:total_energy}
}
\end{equation}

\noindent
where $C_{loss} = 3 \times 10^{19} \times 10^{-3} e$.

% --------------------------------------------------
\subsubsection{Plasma Current $I_p$}

Integrating the Maxwell-Ampère equation over the whole plasma cross-section, and using Stokes' theorem, one gets:

\begin{equation*}
\int_\mathcal{S} (\nablabf\times \Bbf) \cdot \mathbf{dS} = 
\oint_\mathcal{C} \mathbf{B} \cdot \mathbf{d\ell}
= \mu_0 \int_\mathcal{S} \jbf \cdot \mathbf{dS} = \mu_0 I_p
\end{equation*}

Neglecting the elongation of the cross-section, the equation can be approximated as follows:
$$
2\pi a B_p = \mu_0 I_p
$$

\noindent
with $B_p$ the poloidal component of the magnetic field at the separatrix. In the limit of large aspect ratios, it can be easily related to the safety factor $q_a$ at the separatrix\footnote{Actually, $q_a$ is usually taken slightly inside the separatrix, at 95$\%$ of the poloidal magnetic flux.}:

\begin{equation*}
q_a \doteq \frac{a}{R} \frac{B_t}{B_p} 
\end{equation*}

\noindent
with $B_t$ the toroidal component of the magnetic field at the magnetic axis. Then it comes:

\begin{equation}
\boxed{\hat I_p = C_I \frac{\varepsilon^2}{q_a} \; R B}
\label{eq:plasma_current}
\end{equation}

\noindent
where $\hat I_p$ is the current expressed in MA, and $C_I = 2\pi\, 10^{-6} /\mu_0 = 5$  ($\mu_0 = 4\pi\, 10^{-7} \si{H.m^{-1}}$).
Notice that $B_t$ has been safely replaced by $B$ since $B_p\ll B_t$ in tokamak plasmas. 

% --------------------------------------------------
\subsubsection{Power Balance: the link between $Q$ and $\tau_e$}



Using the definition of $Q$ (\ref{eq:definition_Q}) in (\ref{eq:definition_net_power}) leads to\footnote{It would not be useful to replace the expression of $P_{fus}$ in (\ref{eq:definition_Q}) by (\ref{eq:FusionPower1}), since we are looking for a second independent equation. }:

\begin{equation}
P_H = \frac{P_{fus}}{\lambda} + \frac{P_{fus}}{Q} = P_{fus} \frac{1 + Q/\lambda}{Q}
\end{equation}

At equilibrium\footnote{Should the plasma not be at equilibrium and/or be subject to significant radiative losses in the confined region, then $P_{H}$ should be replaced by $(P_{H}-dW/dt-P_{rad})$. The subtraction of $P_{rad}$ ensures the balance equation to be consistent with the retained definition of $\tau_E$.}, i.e. from (eq.\ref{eq:power_balance_simplified}), one has:

\begin{equation}
P_{fus} \frac{1 + Q/\lambda}{Q} = P_{loss}
\end{equation}

\noindent
which can be related to the confinement time $\tau_E$ from its definition  (\ref{eq:definition_confinement_time}):

\begin{equation}
P_{fus} \frac{1 + Q/\lambda}{Q} = \frac{ W }{ \tau_E }
\label{eq:link_btw_Q_and_tau_E}
\end{equation}

The confinement time can be expressed from the scaling law (\ref{eq:scaling_law_IPB98(y,2)}), but the plasma energy $W$ remains to be expressed.


% --------------------------------------------------
\subsubsection{Relationship 2}
Using equations (\ref{eq:scaling_law_IPB98(y,2)}) and (\ref{eq:total_energy}) in (\ref{eq:link_btw_Q_and_tau_E}) leads to\footnote{Remind to integrate the parameter $\kappa^{0.78}$ which has been removed from the confinement time scaling law into the volume constant $C_V$ (as $\kappa^{-0.78}$)}:
\begin{equation}
	\frac
	{ C_{loss} C_V \varepsilon^2  \hat n \hat T R^3 }
	{C_{SL} \hat M^{0.19}  \varepsilon^{0.58} 
		\hat n^{0.41} \hat I_p^{0.93} R^{1.97} B^{0.15}  \hat P_{net}^{-0.69} }
	=
	P_{fus} \frac{1 + Q/\lambda}{Q}
\end{equation}

This is the second equation relating the target $(P_{fus}, Q)$ to the machine parameters $(R,B)$, at condition to express all parameters such as plasma current and density as functions of parameters $(R,B)$. However, we will first discuss the various physical constraints.

% --------------------------------------------------
% --------------------------------------------------
% --------------------------------------------------
\section{Physical Constraints}
% --------------------------------------------------
% --------------------------------------------------
\subsection{Reactivity and Temperature range}
Bosh-Hale tabulation

In the temperature range $T_{\mathrm{[keV]}} \in [10.3-18.5]$ keV, it turns out that the reactivity $\left< \sigma v \right>_{DT}$ can well (with about 10$\%$ error) be approximated by \cite[(1.5.4)]{Wesson2004}: 

\begin{equation*}
\left< \sigma v \right>_{DT} \approx 1.18\, 10^{-24}\; \hat T^2 \;\si{\left[m^3 s^{-1}\right]}
\end{equation*}

% --------------------------------------------------
% --------------------------------------------------
\subsection{MHD Instabilities}
The plasma beta $\beta$ is defined as the ratio of the plasma pressure $p=2nk_BT$ (the factor 2 comes from the fact that the total temperature is considered, $i.e.$ $n(T_e+T_i)$, with the assumption of equal ion and electron temperatures) to the magnetic pressure $B^2/2\mu_0$:
\begin{equation}
\boxed{\beta_\% \doteq \frac{p}{B^2/2\mu_0}
	= C_\beta \frac{\hat n \hat T}{B^2}}
\label{eqn:beta}
\end{equation}
where $\beta_\%=10^2\, \beta$ is expressed in $\%$ and $C_\beta = 4.\,10^2\mu_0\times 10^{19}\times 10^3 e \approx 0.805$.

Several modes (such as e.g. kink, tearing or ballooning modes) become MHD unstable above certain thresholds of pressure gradient and plasma current, so that one can expect that $\beta$ will be subject to a stability limitation which will likely depend on the plasma current. \emph{``However, the concept of $\beta$ limit is not precise. Stability depends on profiles, and any optimisation introduces the questions of which modes of instability to include and what mode numbers to allow. Furthermore, there is uncertainty as to the severity of the nonlinear consequences of the various modes. Nevertheless the intrinsic usefulness of a concise analytic $\beta$ limit in assessing possible tokamak performance has prompted a number of investigations''} \cite{Wesson2004}.

It turns out that the $\beta$ limit, i.e. the maximum stable $\beta$, scales approximately like $\varepsilon/q_a$, which can be recast as $(\mu_0/2\pi)\; I_p/aB$ from the above relations. We shall define $\beta_m$ as follows:
\begin{equation*}
\beta \leqslant \beta_m \doteq g\; \frac{\hat I_p}{a B}
\end{equation*}
The coefficient of proportionality $g$ depends on the considered instabilities. The so-called "Troyon limit" \cite{Troyon1984} puts this coefficient to $2.8\, 10^{-2}$ (or 2.8 if $\beta$ is expressed in $\%$).
It is usual to introduce the normalised $\beta$, called $\beta_N$, defined by \cite[eq.(13.146)]{Freidberg2007}:
\begin{equation}
\beta \doteq \beta_N \frac{\hat I_p}{a B}
\end{equation}
Then, the stability limit simply reads $\beta_N <g$.




% --------------------------------------------------
% --------------------------------------------------
\subsection{Density Limit}

As stated in a recent topical review on the subject \cite{Greenwald2002}, \emph{``in addition to the operational limits imposed by MHD stability on plasma current and pressure, an independent limit on plasma density is observed in confined toroidal plasmas. [...] In tokamaks, [...] there is strong evidence linking the limit to physics near the plasma boundary [...]''}. As a matter of fact, the so-called Greenwald density $n_G$ is not a sharp density limit, since discharges with peaked density profiles can well operate above this value. So far, there is no widely accepted, first principles model for the density limit. Yet, the focus is currently either on mechanisms which lead to strong edge cooling, or on collisionality enhanced turbulent transport.

From \cite[eq.(14.146)]{Freidberg2007}, the most common empirical scaling for (line-averaged) density limit is the following:
\begin{equation}
\boxed{\hat n_G \doteq C_n \frac{\hat I_p}{\varepsilon^2 R^2}  }
\label{eqn:greenwald_density}
\end{equation}
with $C_n = 10/\pi \approx 3.18$.
Using the expression of the plasma current given above, and replacing $\mu_0$ by its value, $\hat n_G$ can be recast as follows:
\begin{equation*}
\hat n_G = C_nC_I \frac{B}{q_aR}
\end{equation*}
Finally, one introduces the normalised density $n_N\doteq n/ n_G$, so that:
\begin{equation}
\hat n = C_nC_I\; n_N\; \frac{B}{q_aR}
\label{eq:n_nN}
\end{equation}



% --------------------------------------------------
% --------------------------------------------------
% --------------------------------------------------
\section{Expression Reformulations}
% --------------------------------------------------
\subsection{Normalizations}
We recommend using the following alternative units: 
\begin{itemize}
    \item $\hat n$ is the density in $10^{19} \si{m^{-3}}$: 
    $\hat n = 10^{-19}\,n_{\si{[m^{-3}]}}$
    \item $\hat T$ is the temperature in $keV$: $\hat T = 10^{-3}\, k_B T_{[K]}\,/e$ \\(with $k_B \approx 1.3807\, 10^{-23} \si{J.K^{-1}}$ the Boltzmann's constant and $e\approx 1.6022\, 10^{-19}C$ the elementary electric charge)
    \item $\hat I_p$ is the plasma current in MA: $\hat I_p = 10^{-6}\,I_{p\,[A]}$
    \item $\hat P$ is the power in MW: $\hat P = 10^{-6}\, P_{[W]}$
    \item $\hat M$ is the mass in Atomic Mass Unit
\end{itemize}



% -------------------------------------------------------------
\subsection{Greenwald Density}
As stated in a recent topical review on the subject \cite{Greenwald2002}, \emph{``in addition to the operational limits imposed by MHD stability on plasma current and pressure, an independent limit on plasma density is observed in confined toroidal plasmas. [...] In tokamaks, [...] there is strong evidence linking the limit to physics near the plasma boundary [...]''}. As a matter of fact, the so-called \textit{Greenwald density} $n_G$ is not a sharp density limit, since discharges with peaked density profiles can well operate above this value. So far, there is no widely accepted, first principles model for the density limit. Yet, the focus is currently either on mechanisms which lead to strong edge cooling, or on collisionality enhanced turbulent transport.

From \cite[eq.(14.146)]{Freidberg2007}, the most common empirical scaling for (line-averaged) density limit is the following:
\begin{equation}
\boxed{\hat n_G \doteq C_n \frac{\hat I_p}{\varepsilon^2 R^2}  }
\label{eq:greenwald_density}
\end{equation}
with $C_n = 10/\pi \approx 3.18$.



% --------------------------------------------------
% --------------------------------------------------
% --------------------------------------------------
\section{Searching for a solution}



% --------------------------------------------------
% --------------------------------------------------
% --------------------------------------------------
\section{Additional discussions}
The model which has been derived can be made more realistic, taking into account the following additional parameters:
\begin{itemize}
	\item peaking factor in the parabolic profile
	\item dilution factor $f_D$ $n_D = n_T = f_D \left( \frac{n}{2} \right)$. 
	It represents the decrease of the available fuel due to non deuterium and tritium plasma species (impurities).
	\item pulsed or steady-state device ? This needs discussion on bootstrap current, current density profile
	\item Radiation losses
\end{itemize} 
